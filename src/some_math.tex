\documentclass{article}


\usepackage{amsmath}
\usepackage[a4paper, total={6in, 8in}]{geometry}

\title{Some math}

\begin{document}

    \section{Vector basics}

    Magnitude of a vector $\vv{v}$ is denoted $|\vv{v}|$. It is the square root of the sum of the squares of a vector

    \begin{align}
        \vv{v} & = {x, y} \\
        |\vv{v}| & = \sqrt{x^2 + y^2}
        \label{eqn:magnitudeOfAVector}
    \end{align}

    The magnitude $|\vv{v}|$ of a one element vector is the square root of the square. It is therefore the same as the absolute value.

    The magnitude $|\vv{v}|$ $=$ norm $=$ modulus $=$ length of a vector.

    A unit vector is a vector with magnitude $=$ 1. To obtain a unit vector in the direction of a vector $\vv{v}$ we divide each element in the vector by the magnitude/length/modulus of the vector. The unit vector is denoted with a “hat”

    \[ \vv{v} = \frac{\vv{v}}{|\vv{v}|}\]


    \section{Polar Coordinates and complex numbers}

    Complex numbers have two components, a real and imaginary. The imaginary part is denoted $i$. Just like the "unit"
    of real numbers is 1, the unit of imaginary numbers is the square root of -1.

    \begin{equation}
        i = \sqrt{-1}
    \end{equation}

    The way complex numbers are often represented is $z=a+bi$. However you can also represent them in polar coordinates

    \[z=r\cdot\cos(\theta) + i r\cdot\sin(\theta) = r\cdot(\cos(\theta) + i\sin(\theta))\]

    The abbeviated form is

    \[z=r\cdot cis(\theta)\]

    where
    \begin{itemize}
        \item $r = $ radius $=$ $\sqrt{x^2 + y^2}$, where $x = $ width and $y =$ height.
        \item $\theta =$ angle $ \tan^{-1}(\frac{y}{x})$
    \end{itemize}


    \section{Exponential form of a complex number}
    If a complex number $z$ has modulus $r$ and argument $\theta$, then

    \[z=r(\cos(\theta) + i \sin(\theta))\]

    Therefore, complex numbers can be represented with only the modulus $r$ and the argument $\theta$. The
    exponential form of a complex number is a simpler way of representing the same thing, using the same
    parameters. To understand, look at Euler's identity equation.

    \[e^{i\theta} = \cos(\theta) + i \sin(\theta) \]

    \section{Fourier transformation}

    \section{Ptychography}
    \begin{itemize}
        \item Lenseless imaging technique. I.e. no need for a lense between equipment and sample.
        \item Instead of using a lense, we can retrieve an image algorithmically. \bf{Phase retrieval algorithms}.
        \item X-ray beam at one side goes through a `limiter' which limits illumination to parts of the sample.
        \item The sample is movable. The diffraction pattern can be measured on a detector, like ours.
        \item Each point of the sample has its own diffraction pattern.
        \item When moved around the sample, care is taken such that each point of the sample overlaps with the
              point next door. This overdetermines the algorthm: the redundant information in overlapping samples
              is critical for the algorithm to accurately produce imaging result.
        \item I am not yet able to find a mathematical description of what the algorithm is calculating.
        \item
        \item Maurtiz and dmitry are handling high level 4D stemideas
        \item I've been working on testing, scan box and learning about libertem.
        \item Rick is away
        \item Bram is working on... something.
        \item I cannot remember what Siamack is working on .
        \item
        \item CRoss masking - because they get higher counts. Mask center. 

    \end{itemize}

\end{document}
