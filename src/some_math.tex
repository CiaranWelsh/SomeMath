\documentclass{article}


\usepackage{amsmath}
\usepackage[a4paper, total={6in, 8in}]{geometry}

\title{Some math}

\begin{document}

    \section{Vector basics}

    Magnitude of a vector $\vv{v}$ is denoted $|\vv{v}|$. It is the square root of the sum of the squares of a vector

    \begin{align}
        \vv{v} & = {x, y} \\
        |\vv{v}| & = \sqrt{x^2 + y^2}
        \label{eqn:magnitudeOfAVector}
    \end{align}

    The magnitude $|\vv{v}|$ of a one element vector is the square root of the square. It is therefore the same as the absolute value.

    The magnitude $|\vv{v}|$ $=$ norm $=$ modulus $=$ length of a vector.

    A unit vector is a vector with magnitude $=$ 1. To obtain a unit vector in the direction of a vector $\vv{v}$ we divide each element in the vector by the magnitude/length/modulus of the vector. The unit vector is denoted with a “hat”

    \[ \vv{v} = \frac{\vv{v}}{|\vv{v}|}\]


    \section{Polar Coordinates and complex numbers}

    Complex numbers have two components, a real and imaginary. The imaginary part is denoted $i$. Just like the "unit"
    of real numbers is 1, the unit of imaginary numbers is the square root of -1.

    \begin{equation}
        i = \sqrt{-1}
    \end{equation}

    The way complex numbers are often represented is $z=a+bi$. However you can also represent them in polar coordinates

    \[z=r\cdot\cos(\theta) + i r\cdot\sin(\theta) = r\cdot(\cos(\theta) + i\sin(\theta))\]

    The abbeviated form is

    \[z=r\cdot cis(\theta)\]

    where
    \begin{itemize}
        \item $r = $ radius $=$ $\sqrt{x^2 + y^2}$, where $x = $ width and $y =$ height.
        \item $\theta =$ angle $ \tan^{-1}(\frac{y}{x})$
    \end{itemize}


    \section{Exponential form of a complex number}
    If a complex number $z$ has modulus $r$ and argument $\theta$, then

    \[z=r(\cos(\theta) + i \sin(\theta))\]

    Therefore, complex numbers can be represented with only the modulus $r$ and the argument $\theta$. The
    exponential form of a complex number is a simpler way of representing the same thing, using the same
    parameters. To understand, look at Euler's identity equation.

    \[e^{i\theta} = \cos(\theta) + i \sin(\theta) \]

    \section{Fourier transformation}

\end{document}
